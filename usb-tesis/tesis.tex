\documentclass[letterpaper, 12pt, oneside]{tesis}

% Paquetes para idioma
\usepackage[spanish]{babel}
\usepackage[utf8]{inputenc}
\usepackage[fixlanguage]{babelbib}

% Otros paquetes instalados
% Básicos
\usepackage{natbib}
\usepackage{enumerate}

% Para dibujar figuras
\usepackage{tikz}

% Para cambiar el color de las letras
\usepackage{color}

% Para incluir código (básico)
\usepackage{verbatim}
\usepackage{fancyvrb}

% Para incluir hipervínculos
\usepackage{hyperref}
\hypersetup{urlcolor=blue, colorlinks=false}

% Para más símbolos matemáticos
\usepackage{amsmath}
\usepackage{amsthm}
\usepackage{amssymb}

% Para colocar teoremas en cajas
\usepackage{mdframed}

% Para texto Lorem Ipsum
\usepackage{blindtext}

% Paquetes locales
% Puedes agregar paquetes locales (archivos .sty) en un subdirectorio % 'paquetes'.
% Utiliza la sintaxis \usepackage{paquetes/nombrePaquete}

% Todas las imágenes se cargan del subdirectorio 'img' por defecto.
\graphicspath{{img/}}

% Sangrías de 3 espacios (3 veces el espacio de la x)
\parindent 3ex 

% Interlineado
\setlength{\baselineskip}{1.5pt}

% Interpárrafo
\setlength{\parskip}{16.5pt}

\topmargin 2cm

\renewcommand{\tablename}{Tabla}
\newcommand\listsymbolname{Acrónimos y Símbolos}

\begin{titlepage}
    \title{\vspace{-2cm} \includegraphics[width=1.2in]{./usb.png} \\[.2cm]
        \large Universidad Simón Bolívar \\
        Decanato de Estudios Profesionales \\
        Coordinación de Ingeniería de la Computación
        \vfill \LARGE @títuloProyecto \vfill}
    \author{Por: \\
        @autor1 \\
        @autor2 \\[1.2cm]
        Realizado con la asesoría de: \\
        @tutor \\[1.2cm]
        PROYECTO DE GRADO \\
Presentado ante la Ilustre Universidad Simón Bolívar \\
como requisito parcial para optar al título de \\
Ingeniero de Computación}
    \date{Sartenejas, @mes de @año}
\end{titlepage}

\begin{document}
\frontmatter
\maketitle
\setstretch{1.3}

% Se incluye el acta de evaluación, verificar que se corresponda
% con el formato aceptado actualmente por el Decanato.
% Pagina del acta final
\begin{titlepage}
\begin{center}

% Upper part
\includegraphics[scale=0.5]{usb.png} \\

\textsc {\large UNIVERSIDAD SIMÓN BOLÍVAR} \\
\textsc{DECANATO DE ESTUDIOS PROFESIONALES\\
COORDINACIÓN DE INGENIERÍA DE LA COMPUTACIÓN}

\bigskip
\bigskip
\bigskip
\bigskip
\bigskip
\bigskip

% Title
\textsc{ACTA FINAL PROYECTO DE GRADO}

\bigskip
\bigskip

\textsc{\bfseries @títuloProyecto}

\bigskip
\bigskip
\bigskip
\bigskip

\begin{minipage}{\textwidth}
\centering
Presentado por: \\
\textsc{\bfseries @autor1} \\
\textsc{\bfseries @autor2} \\

\bigskip
\bigskip
\bigskip

Este Proyecto de Grado ha sido aprobado por el siguiente jurado examinador: \\

\bigskip
\bigskip

% Despues de cada line coloca el (los) nombre(s) de
% cada uno de los integrantes del jurado.
\line(1,0){200} \\
@tutor\\

\bigskip
\bigskip

\line(1,0){200} \\
@jurado1\\

\bigskip
\bigskip

\line(1,0){200} \\
@jurado2\\
\end{minipage}

\bigskip
\bigskip
\vfill

% Date/Fecha
{\large \bfseries Sartenejas, @día de @mes de @año}

\end{center}
\end{titlepage}
 

% El resumen debe ser de una sola página
\addtotoc{Resumen}
\abstract{
\addtocontents{toc}{\vspace{1em}}
\blindtext

\blindtext

% Las palabras clave son generalmente los nombres de áreas de investigación a
% los cuales está asociado el trabajo. Generalmente son tres o cuatro.
\noindent \begin{small} \textbf{Palabras clave}: @palabra1, @palabra2, @palabra3.
\end{small}

% Iniciar nueva página luego del resumen
\clearpage
\setstretch{1.3}

% Agradecimientos
\acknowledgements{
\addtocontents{toc}{\vspace{1em}}
\blindtext

\blindtext
}
\clearpage

\pagestyle{fancy}

% Tabla de contenidos o índice
\lhead{\emph{Índice General}}
\tableofcontents

% Estos índices solamente se usan si el libro contiene figuras, tablas y
% algoritmos. Si alguno de estos no se utiliza, comentar o eliminar las líneas
% pertinentes.
\lhead{\emph{Índice de Figuras}}
\listoffigures

\lhead{\emph{Índice de Tablas}}
\renewcommand*\listtablename{Índice de Tablas}
\listoftables

%\lhead{\emph{Índice de Algoritmos}}
%\renewcommand*\listalgorithmname{Índice de algoritmos}
%\listofalgorithms

\setstretch{1.5}
\clearpage
\lhead{\emph{Acrónimos y símbolos}}
\listofsymbols{ll}
{

    % Aquí van las siglas
    \textbf{SIGLAS} & \textbf{S}iglas \textbf{I}sla \textbf{G}rafo 
                      \textbf{L}aos \textbf{A}ve \textbf{S}erpiente\\
    \textbf{ACM} & \textbf{A}ssociation for \textbf{C}omputing \textbf{M}achinery\\
    &\\
    \hline
    &\\

    % Aquí van los símbolos
    $\iff$ & doble implicación, si y sólo si\\
    $\Rightarrow$ & implicación lógica\\
    $[u:=v]$ & sustitución textual de $u$ por $v$
}

%% ----------------------------------------------------------------
% End of the pre-able, contents and lists of things
% Begin the Dedication page

\setstretch{1.3}  % Return the line spacing back to 1.3

\pagestyle{empty}  % Page style needs to be empty for this page

\dedicatory{
    \textbf{Dedicatoria} \bigskip

    A @personasImportantes, por @razonesDedicatoria.
}

\addtocontents{toc}{\vspace{2em}}

\mainmatter
\pagestyle{fancy}

% Se incluye el cuerpo de la tesis en este documento.

\chapter*{Introducción}
\label{intro}
\lhead{\emph{Introducción}}
\addcontentsline{toc}{chapter}{Introducción}

% Descripción del problema, de lo general hacia lo específico
\blindtext 

% Trabajos anteriores
\blindtext
\cite{fuente1} presenta un trabajo que \ldots

\citeauthor{fuente2} es otro autor que \ldots

% Objetivo general
\blindtext

% Objetivos específicos
\blinditemize

Cachucha\footnote{Lorem Ipsum.}.

% Organización del trabajo
% Se describe brevemente qué se hace en cada capítulo
\blindtext[4]


% El número de capítulos varía. En mi libro fueron cuatro (sin contar
% introducción y conclusión).
\chapter{@nombreCapítulo}
\label{capitulo1}
\lhead{Capítulo 1. \emph{@nombreCapítulo}}

% De qué va a tratar el capítulo
% El capítulo 1 suele ser el marco teórico.

\section{@sección}
\begin{definition}
\label{definicion1}
\blindtext, donde:
\begin{itemize} 
\item $X$ es $\gamma - 2$.
\item $A$ es un conjunto de \textbf{cosas}.
\end{itemize}
\end{definition}

La Figura \ref{usb} muestra el símbolo de nuestra universidad.
\begin{figure}[h!]
\centering
\includegraphics[width=0.4\textwidth]{usb.png}
\caption[La popular \textit{cebolla}]{La popular \textit{cebolla}, símbolo de la USB.}
\label{usb}
\end{figure}

\begin{verbatim}
para escribir código
    básico
\end{verbatim}
\Blindtext

\begin{Verbatim}[commandchars=\\\{\}, codes={\catcode`$=3\catcode`^=7}]
var x = 21;
if (esto_es_código) {
    imprimir(foo);
}
(lisp (listas (?paréntesis))
\end{Verbatim}

\blindtext
\subsection{@subSección}
\Blindtext

\section{@sección}
\blindtext

\begin{mdframed}
\begin{theorem}
\label{principal}
\textbf{Propiedades formales}
\blindenumerate
\end{theorem}
\end{mdframed}

\subsection{@subsección}
\subsubsection{@subsubsección}
\blindenumerate

La Figura \ref{grafo} lo muestra.

\shorthandoff{<>."}
\begin{figure}[h]
\begin{center}
\begin{tikzpicture}[shorten >=1pt, thick]%[shorten >=1pt,node distance=2cm,>=stealth',thick]
  \node [shape=circle,fill=black,inner sep=1.5pt,label=below:$s$] (q0) at (0,0) {};
  \node [shape=circle,fill=black,inner sep=1.5pt,label=below:$1$] (q1) at (2,0) {};
  \node [shape=circle,fill=black,inner sep=1.5pt,label=below:$t$] (q2) at (4,0) {};
  \path[->] (q0) edge (q1) (q1) edge (q2);
\end{tikzpicture}
\end{center}
\caption[@descripcionCorta]{@descripcionLarga}
\label{grafo}
\end{figure}

\begin{enumerate}[--]
\item 1
\item 2
\item 3
\end{enumerate}

\begin{tabular}{ll}
1 & 2\\ \hline
1 & 2\\
1 & 2\\
\end{tabular}

\blindtext. Tabla \ref{tabla:resultados}.

\begin{figure}[h]
\begin{alignat*}{1}
A\   & \longrightarrow B \mid C\\
\end{alignat*}
\caption[Gramática]{Gramática de un lenguaje.}
\label{gram}
\end{figure}

\blindtext
\begin{table}[h!]
\begin{center}
\begin{tabular}{llllll}
\multicolumn{4}{@{}c}{Nombre del experimento} \\
\midrule
              &    éxitos/intentos & tiempo (ms) & espacio (kB) \\
\midrule
instancia1          &        28/30 &    23 &       1.7 \\
instancia2          &        50/70 &    12 &       32.7 \\
\midrule
\end{tabular}
\end{center}
\caption[Resultados X/Y]{Resultados de X para Y}
\label{tabla:resultados}
\end{table}

\begin{equation}
\label{eq}
\Phi = (\forall x) (R x)
\end{equation}

En el Apéndice \ref{apendiceA} se encuentra.

\chapter{@nombreCapítulo}
\label{capitulo2}
\lhead{Capítulo 2. \emph{@nombreCapítulo}}

\Blindtext

\chapter{@nombreCapítulo}
\label{capitulo3}
\lhead{Capítulo 3. \emph{@nombreCapítulo}}

\Blindtext

\chapter{@nombreCapítulo}
\label{capitulo4}
\lhead{Capítulo 4. \emph{@nombreCapítulo}}

\Blindtext


\chapter{@conclusionesYRecomendaciones}
\label{conclusiones}
\lhead{\emph{@conclusionesYRecomendaciones}}

\Blindtext

% Incluir recomendaciones para trabajos futuros
\blindenumerate


% El estilo de la bibliografía es AAAI, definido en el archivo aaai.bst.
\label{Bibliography}
\bibliography{bibliografia}
\lhead{\emph{Bibliografía}}
\bibliographystyle{aaai}
\addtocontents{toc}{\vspace{2em}}

% Apéndices
\appendix
\chapter{@nombreApendice}
\label{apendiceA}
\lhead{Apéndice A. \emph{@nombreApendice}}

% En los apéndices se incluye cualquier información que no sea esencial para la
% comprensión básica del trabajo, pero provea ejemplos y casos de estudio
% extendidos que permitan un análisis más exhaustivo.

\section{@sección}
\blindtext

\subsection{@subsección}
\Blindtext

``Saludo''.

\begin{figure}[h!]
\centering
\includegraphics[width=0.5\textwidth]{grafo.pdf}
\caption[Grafo]{Grafo gris.}
\label{imagen:grafo}
\end{figure}

\begin{figure}[h!]
\centering
\includegraphics[width=\textwidth]{grafocolor.pdf}
\caption[Grafo coloreado (esto sale en la tabla de contenidos)]{Grafo con color.}
\label{imagen:grafodecolores}
\end{figure}

\chapter{@nombreApendice}
\label{apendiceB}
\lhead{Apéndice B. \emph{@nombreApendice}}

\Blindtext

\addtocontents{toc}{\vspace{2em}}

\backmatter

\end{document}
